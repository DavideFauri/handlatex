% \iffalse
%% File: hand.dtx
%% Copyright (C) 2008 Sylvain Fourmanoit <syfou@users.sourceforge.net>
%%
%% This code is released under the terms of the GNU General Public License
%% (GPL), version 2. You should have received a complete copy of the licence
%% along the software: see COPYING.
%%
%% Permission is hereby granted, free of charge, to any person obtaining a copy
%% of this software and associated documentation files (the 'Software'), to deal
%% in the Software without restriction, including without limitation the rights
%% to use, copy, modify, merge, publish, distribute, sublicense, and/or sell
%% copies of the Software, and to permit persons to whom the Software is
%% furnished to do so, subject to the following conditions:
%%
%% The above copyright notice and this permission notice shall be included in
%% all copies of the Software and its documentation and acknowledgment shall be
%% given in the documentation and software packages that this Software was used.
%%
%% THE SOFTWARE IS PROVIDED 'AS IS', WITHOUT WARRANTY OF ANY KIND, EXPRESS OR
%% IMPLIED, INCLUDING BUT NOT LIMITED TO THE WARRANTIES OF MERCHANTABILITY,
%% FITNESS FOR A PARTICULAR PURPOSE AND NONINFRINGEMENT. IN NO EVENT SHALL THE
%% AUTHORS BE LIABLE FOR ANY CLAIM, DAMAGES OR OTHER LIABILITY, WHETHER IN AN
%% ACTION OF CONTRACT, TORT OR OTHERWISE, ARISING FROM, OUT OF OR IN CONNECTION
%% WITH THE SOFTWARE OR THE USE OR OTHER DEALINGS IN THE SOFTWARE.
%%
%% The LaTeX part of this package consists of the files hand.dtx, hand.def and
%% hand.ins.
%%
%<hand>\def\handReleaseInfo{2008/17/02, v1.0.0}%
\def\handlowwordangle{-2}%
\def\handmaxparscale{1.25}%
\def\handminparscale{0.8}%
\def\handencoding{utf-8}%
\def\handhighwordangle{2}%
\def\handdriver{latex}%
\def\handminparangle{-2}%
\def\handfreqword{0.4}%
\def\handmaxparangle{2}%
%
%<hand>\ProvidesPackage{hand}[\handReleaseInfo\space Handwriting environment]%
%<hand>\NeedsTeXFormat{LaTeX2e}%
%<hand>\RequirePackage{keyval}%
%<hand>\RequirePackage{rotating}%
%<hand>\RequirePackage{setspace}%
%<*driver>
\documentclass{ltxdoc}
\usepackage[
  bookmarks=true,
  pdfstartview=FitH,
  colorlinks=True,	 % OK, so I probably shoudn't depend on that, but hey -
  urlbordercolor={1 1 1} % It seems to be included de facto on most base LaTeX 
]{hyperref} 		 % environments these days...
\usepackage{amssymb}     % Idem.
\def\handLaTeXRevision{dummy}
\usepackage{hand}
\begin{document}
  \title{The \texttt{hand} package}
  \author{Sylvain Fourmanoit\\ \href{mailto:syfou@users.sourceforge.net}
    {syfou@users.sourceforge.net}}
  \date{\handReleaseInfo\thanks{Latest version available at all time from
        \url{http://code.google.com/p/handlatex/}.}}
  \maketitle
  \DocInput{hand.dtx}
\end{document}
%</driver>
% \fi
% \iffalse
%<*hand>
% \fi
%%
% \newcommand{\comment}[1]{\fbox{\begin{minipage}{.9\linewidth}#1\end{minipage}}}
%
% \begin{abstract}
% The \texttt{hand} package is the \LaTeX\ interface to the special-purpose
% \texttt{handlatex} frontend. It provides a simple environment, \cmd{\handpar},
% that can be used to inject seemingly random variations in placement,
% interlining and orientation of paragraphs, as well as visual word
% shifting. Combined with appropriate fonts, it reasonably mimics the typical
% rendering of a sloppy handwritten text on delinated paper.
% \end{abstract}
%
% \section{Usage}
%
% \subsection{Invokation}
%
% Once installed, the \texttt{hand} package can be included in a \LaTeX\ 
% document preamble with the usual:
% \begin{verbatim}
% \usepackage[]{hand}\end{verbatim}
%
% \subsection{Mandatory \texttt{handlatex} Frontend}
%
% Any document using \texttt{hand} must be compiled through the special-purpose
% \texttt{handlatex} frontend (shipping with this package); this
% frontend will in turn invoke some \LaTeX-capable backend (\texttt{latex},
% \texttt{pdflatex}, etc.), specifiable by the user (see \texttt{driver} option
% below), to generate the final output.\\
%
% Have a look at the provided \texttt{README} if you need further information on
% the frontend. Online help is also available via \texttt{handlatex --help}.
%
% \section{User Interface}
% \begin{macro}{handpar}
%
% The \cmd{\handpar} environment is the only one from the \texttt{hand} package
% you should be using directly; it basically randomizes the appearance of the
% enclosed paragraph to give it a more ``handwritten'' look. Typical use would
% look like:
%
% \begin{verbatim}
% \begin{handpar}
% Lorem ipsum dolor sit amet, feugiat ad metus vestibulum at
% eget. Condimentum est dictumst, quis cras congue et turpis, ornare
% lacinia hendrerit platea in. Nulla est proin pellentesque hendrerit
% nulla vitae.
% \end{handpar}\end{verbatim}
%
% Paragraph and words orientation as well as linespacing will change
% pseudo-randomly, according to package options, as specified below.  A demo
% document using \texttt{xelatex} from \TeX\,Live, \texttt{sample.tex}, is
% distributed along the package.
% \end{macro}
%
% \section{Options}
% Options are key-value pairs, specified when including the \texttt{hand}
% package, like this:
% \begin{verbatim}
% \usepackage[driver=pdflatex, freqword=0.05]{hand}\end{verbatim}
%
% \noindent Options set this way can be overruled directly from the command
% line; each key presented below can be set using a ``gnu-style'' long argument
% passed to \texttt{handlatex}: see \texttt{handlatex --help} for details.
%
% \begin{macro}{driver}
% Used by \texttt{handlatex} to determine which backend to invoke (\texttt{latex},
% \texttt{pdflatex}, \texttt{xelatex}, etc.).\\
% \textbf{Default value}: \texttt{\handdriver}
% \end{macro}
%%
% \comment{
% The \texttt{handlatex} frontend is just a pre-processor: at compilation time,
% the real heavy-lifting will be performed by the selected \texttt{driver},
% automatically called by \texttt{handlatex}. The document should follow whatever
% requirements this \texttt{driver} has.
% }
%
% \begin{macro}{encoding}
% Document character set used for input (\texttt{ascii}, \texttt{latin1},
% \texttt{utf-8}, etc.).\\
% \textbf{Default value}: \texttt{\handencoding}
% \end{macro}
%
% \comment{ If ever the document encoding is not a strict superset of ASCII,
% \texttt{handlatex} might not be able to parse the document at all, let alone
% identify the \texttt{encoding} option! In this case, you can always force
% encoding selection right on the command line: \texttt{handlatex
% --encoding=some\_encoding ...}  }
%
% \begin{macro}{minparangle}
% The minimal integer value in degrees the \cmd{\handpar} environment should
% be rotated by (around its lower left corner, counterclockwize).\\
% \textbf{Default value}: \handminparangle
% \end{macro}
%
% \begin{macro}{maxparangle}
% The maximal integer value in degrees the \cmd{\handpar} environment should
% be rotated by (around its lower left corner, counterclockwize).\\
% \textbf{Default value}: \handmaxparangle
% \end{macro}
%
% \comment{
% The paragrah angle 
% ($ \mathrm{parangle} \in [\,\mathrm{minparangle}, \mathrm{maxparangle}\,] $), 
% follows a simple discrete Markov walk: consecutive \cmd{\handpar}
% environements will be rotated by angles no more different than one degree.
% }
%
% \begin{macro}{minparscale}
% The minimal float value the \cmd{\baselineskip} or the \cmd{\handpar}
% environment should be multiplied by.\\
% \textbf{Defaul value}: \handminparscale
% \end{macro}
%
% \begin{macro}{maxparscale}
% The maximal float value the \cmd{\baselineskip} or the \cmd{\handpar}
% environment should be multiplied by.\\
% \textbf{Defaul value}: \handmaxparscale
% \end{macro}
%
% \comment{
% Paragrah interlining (``scaling'') is a uniform variable where 
% $ \mathrm{parscale} \in [\,\mathrm{minparscale}, \mathrm{maxparscale}]$.
% }
%
% \begin{macro}{lowwordangle}
% The low angle in degree any picked word (see \texttt{freqword} below)
% belonging to the \cmd{\handpar} environment can be rotated around its
% bottom-right corner.\\
% \textbf{Default value}: \handlowwordangle
% \end{macro}
%
% \begin{macro}{highwordangle}
% The high angle in degree any picked word (see \texttt{freqword} below)
% belonging to the \cmd{\handpar} environment can be rotated around its
% bottom-right corner.\\
% \textbf{Default value}: \handhighwordangle
% \end{macro}
%
% \comment{
% Picked words rotation will be either \texttt{lowwordangle} or
% \texttt{highwordangle}, with the same odds.
% }
%
% \begin{macro}{freqword}
% The frequency at which a word belonging to the \cmd{\handpar} environment
% should be ``visual shifted'' using the \texttt{lowwordangle} and
% \texttt{highwordangle} options above. \\
% \textbf{Default value}: \handfreqword
% \end{macro}
%
% \comment{
% in \texttt{handlatex}, word selection is implemented as a mere Bernouilli
% variable (i.e. a sequence of independent Bernouilli trials of probability $
% \mathrm{freqword}$).
% }
%
% \section{Limitations / Bugs}
%
% \begin{itemize}
% \item \cmd{\handpar} parsing in \texttt{handlatex} is pretty limited right
% now: it chokes silently and bails out on any closing brace (``\}'') used
% inside a \cmd{\handpar} environment; this basically means that if you try to
% use any \LaTeX\ command at all inside one of them, you will end up with a
% plain, unstylised paragraph...
%
% \item \texttt{handlatex} is implemented in Python, not as a set of
% \TeX\ macros or in \Web2C\footnote{I tried writing a pure
% \TeX\ implementation for two days... For the life of me, I just couldn't; I
% swear the fact a language is Turing-complete will no longer be a compelling
% enough argument for me to bite the bullet -- so is \texttt{brainfuck}, after
% all.}: from a pure \LaTeX\ point of view, it is bound to be less
% portable.  \end{itemize}
%
% \section{Legalese}
%
% Copyright\copyright\ 2008 Sylvain Fourmanoit
% $<$\href{mailto:syfou@users.sourceforge.net}{syfou@users.sourceforge.net}$>$
% \\
%
%  This documentation and associated code is released under the terms of the
% \href{http://www.gnu.org/licenses/gpl2.html}{GNU General Public License
% (GPL), version 2}. You should have received a complete copy of the licence
% along the software: see \texttt{COPYING}.\\
%
% Permission is hereby granted, free of charge, to any person obtaining a copy
% of this software and associated documentation files (the ``Software''), to
% deal in the Software without restriction, including without limitation the
% rights to use, copy, modify, merge, publish, distribute, sublicense, and/or
% sell copies of the Software, and to permit persons to whom the Software is
% furnished to do so, subject to the following conditions:\\
%
% The above copyright notice and this permission notice shall be included in
% all copies of the Software and its documentation and acknowledgment shall be
% given in the documentation and software packages that this Software was
% used.\\ 
%
% {\sc The software is provided ``as is'', without warranty of any kind,
% express or implied, including but not limited to the warranties of
% merchantability, fitness for a particular purpose and noninfringement. in no
% event shall the authors be liable for any claim, damages or other liability,
% whether in an action of contract, tort or otherwise, arising from, out of or
% in connection with the software or the use or other dealings in the
% software.}
%
% \section{The Code}
% \subsection{Initial imports and checks}
%    \begin{macrocode}
\@ifundefined{handLaTeXRevision}{% 
\PackageError{hand}{handlatex is required to compile this document}}%
%    \end{macrocode}
%
% \subsection{Options handling}
%    \begin{macrocode}
\define@key{interface}{driver}[1]{\def\hand@driver{#1}}%
\define@key{interface}{encoding}[1]{\def\hand@encoding{#1}}%
\define@key{interface}{minparangle}[1]{\def\hand@minparangle{#1}}%
\define@key{interface}{maxparangle}[1]{\def\hand@maxparangle{#1}}%
\define@key{interface}{minparscale}[1]{\def\hand@minparscale{#1}}%
\define@key{interface}{maxparscale}[1]{\def\hand@maxparscale{#1}}%
\define@key{interface}{lowwordangle}[1]{\def\hand@lowwordangle{#1}}%
\define@key{interface}{highwordangle}[1]{\def\hand@highwordangle{#1}}%
\define@key{interface}{freqword}[1]{\def\hand@freword{#1}}%
%%
\newcommand{\Parse@options}{}%
\long\def\Parse@Options#1\@nil{\setkeys{interface}{#1}}%
\DeclareOption*{\expandafter\Parse@Options\CurrentOption\@nil}%
\ProcessOptions%
%%
\newcommand{\hand@check}[2]{\@ifundefined{#1}{\@namedef{#1}{#2}}{}}%
\hand@check{hand@driver}{\handdriver}%
\hand@check{hand@encoding}{\handencoding}%
\hand@check{hand@minparangle}{\handminparangle}%
\hand@check{hand@maxparangle}{\handmaxparangle}%
\hand@check{hand@minparscale}{\handminparscale}%
\hand@check{hand@maxparscale}{\handmaxparscale}%
\hand@check{hand@lowwordangle}{\handlowwordangle}%
\hand@check{hand@highwordangle}{\handhighwordangle}%
\hand@check{hand@freqword}{\handfreqword}%
\PackageInfo{hand}{ %
  driver: \hand@driver, %
  encoding: \hand@encoding, %
  minparangle: \hand@minparangle, %
  maxparangle: \hand@maxparangle, %
  minparscale: \hand@minparscale, %
  maxparscale: \hand@maxparscale, %
  lowwordangle: \hand@lowwordangle, %
  highwordangle: \hand@highwordangle, %
  freqword: \hand@freqword }%
%%
%    \end{macrocode}
% \subsection{The Environments}
%    \begin{macrocode}
\newcommand{\handword}[2]%
{%
\rotatebox[origin=rb]{#1}{#2}%
}%
%%
\newcommand{\hand@parfullbeg}[2]%
{%
\noindent\begin{turn}{#1}\begin{minipage}{\linewidth}%
\begin{spacing}{#2}%
}%
%%
\newcommand{\hand@parfullend}%
{\end{spacing}\end{minipage}\end{turn}}%
%%
\newenvironment{handparfull}[2]%
{\hand@parfullbeg{#1}{#2}}{\hand@parfullend}%
%%
\newenvironment{handpar}%
{\hand@parfullbeg{0}{1}}{\hand@parfullend}%
%    \end{macrocode}
% \noindent $\blacksquare$
%%
% \iffalse
%</hand>
% \fi
%
\endinput
